\documentclass[10pt,a4paper]{book}

\begin{document}

\small

\begin{flushright}
  \textsf{\textbf{Regular Expression}} $|$ \textbf{\textsf{267}}
\end{flushright}

Closure is a property which describes when we combine any two elements of the set; the result is
also included in the set.

If we multiply two integer numbers, we will get another integer number. Since this process is always
true, it is said that the integer numbers are ‘closed under the operation of multiplication’. There is simply
no way to escape the set of integer numbers when multiplying.

Let $S = \{1,2,3,4,5,6,7,8,9,10….\}$ is a set of integer numbers.

$1 × 2 = 2$

$2 × 3 = 6$

$5 × 2 = 10$

All are included in the set of integer numbers.

We can conclude that integer numbers are closed under the operation of multiplication.

\begin{flushleft}
  \textbf{Theorem 5.3}: Two REs $L_{1}$ and $L_{2}$ over $\Sigma$ are closed under union operation.

  \textrm{\textbf{Proof:}} We have to prove that if $L_{1}$ and $L_{2}$ are regular over $\Sigma$,then their union, i.e.,$L_{1} \cup L_{2}$ will be also regular.

As $L_{1}$ and $L_{2}$ are regular over $\Sigma$, there must exist FA $ M_{1} = (Q_{1}, \Sigma, \delta_{1}, q_{01}, F_{1}) $ and $ M_{2} = (Q_{2}, \Sigma , \delta_{2},q_{02}, F_{2})$ such that $L_{1} \varepsilon M_{1}$ and $L_{2} \varepsilon M_{2}$

Assume that there is no common state between $Q_{1}$ and $Q_{2}$, i.e., $Q_{1} \cap Q_{2} = {\O}$.

Defi ne another FA, $M_{3} = (Q, \Sigma, \delta, q_{0}, F)$ where

1. $Q = Q_{1} \cup Q_{2} \cup \{q_{0}\}$, where $q_{0}$ is a new state $\notin Q_{1} \cup Q_{2}$

2. $F = F_{1} \cup F_{2}$

3. Transitional function $\delta$ is defi ned as $\delta (q_{0}, \varepsilon) \rightarrow \{q_{01}, q_{02}\}$

and\qquad\qquad\qquad\qquad\qquad $\delta(q, \Sigma) \rightarrow \delta_{1}(q, \Sigma) \:\textrm{if}\: q \varepsilon Q_{1}$

$$\delta(q, \Sigma) \rightarrow \delta_{2}(q, \Sigma) \:\textrm{if}\: q \varepsilon Q_{2}$$

\end{flushleft}

It is clear from the previous discussion that from $q_{0}$ we can reach either the initial state $q_{1}$ of $M_{1}$ or the initial state $q_{2}$ of $M_{2}$.

Transitions for the new FA, M, are similar to the transitions of $M_{1}$ and $M_{2}$.

As $F = F_{1} \cup F_{2}$, any string accepted by$M_{1}$or$M_{2}$ will also be accepted by M.

Therefore, $L_{1} \cup L_{2}$ is also regular.

\emph{\textbf{Example:}}

Let\qquad\qquad\qquad\qquad\qquad\; $L_{1} = a^{*}(a + b)b^{*}$
$$L_{2} = ab(a + b)b^{*}$$

The FA $M1$ accepting $L1$ is as shown in Fig. $5.38(a)$.

The FA $M2$ accepting $L2$ is as shown in Fig. $5.38(b)$.

The machine M produced by combining$M1$and$M2$is as shown in Fig. $5.38(c)$.

It accepts $L_{1} \cup L_{2}$.

\emph{\textbf{Theorem 5.4}}: The complement of an RE is also regular.

If L is regular, we have to prove that $L^{T}$ is also regular.

\qquad As L is regular, there must be an FA, $M = (Q, \Sigma, \delta, q_{0}, F)$, accepting L.

M is an FA, and so M has a transitional system.

\quad Let us construct another transitional system $M\prime$ with the state diagram of M but reversing the direction of the directed edges. $M\prime$ can be designed as follows:

1. The set of states of $M\prime$ is the same as M

2. The set of input symbols of $M\prime$ is the same as M















\end{document}

\documentclass[10pt,a4paper]{book}

\begin{document}

\small

\begin{flushright}
NET-BASED DISSEMINATION OF E-RESEARCH RESULTS\quad \textbf{187}
\end{flushright}

\begin{flushleft}
\normalsize
\textbf{CREATIONG QUALITY CONTENT}
\end{flushleft}


There are many guides to academic writing that describe the process and form of dissemination your findings. Rather than focus on these more generic skills,we look at the diverse ways in which writing for the Net is unique and the multitude of ways in which the Net can be used to disseminate the results of your work.

Research has shown that we process information from the screen in ways that are different form the way we read texts or paper content (Kanuka $\&$ Szabo,1999).Net readers are more likely to skim rather than read meticulously through screen presented content. Thus,e-reserchers should use the techniquse of the newspaper editor, rather than the novelist to present their findings in a Webdocument. For example,the style of screen formatted materials should make extensive use of headings,bolding,numbered and bulleted lists,keywords,andhighlighting,and effective use of white space.These formatting techniques allow the reader to focus on items of partieular interest and skim through that which is not of interest.For these same reasons,the content should be concise and to the point.

some experts suggest suggest using the inverted pyramid style of presentation that was developed by newspaper editors and reporters. Unlike traditional research papers that begin with an unresolved problem, then present all the relevant past research and methodology,and finally conclude with results and applications,aninverted pyramid style begins with the most important and relevant content. Less relevant and more detailed content is placed at, or near, the end of the article.Inclidentally,this style not only allows busy readers the capacity to stop reading at any time knowing they have already convered the most relevant material,but it also provides the writer or editor the capacity to omit content from the end of the article when space problems arise.While the space element dose not apply to the Web(with the exception of server space),the psychological benefits of brevity remain as relevant as ever.

Readers of Web documents are also less impressed with superfluous,hypermarketing text common in television ads,posters,and flyers. Web readers want the facts and want to believe that you are telling them the unbiased results of your research.In an empirical study of these techniques,Morkes and Nielsen(1998)found significant improvements in time to read, errors in recall,and overall stisfaction between content formatted for the Web versus a technical reportformatted for paper presentation.

Despite the lage number of prescriptive guidelines and articles for Web writing (for example see Introduction to Hypertext Writing Style at http://www.bu.edu/cdaly
/hyper.html),we are also aware that the nature of the Web,and readers'apprroach to screen reading,is changing.New technologies(including electronic paper and very high resolution screens)as well as evidence of successfull Net publication using a variety of writing formats and styles,remind us that the Net thrives on diversity and that there is no single formula for all forms of effective research results dissemination(Bresler,2000).

The e-resercher's goal is to work the content into a form that is clearly and easily understood by the intended audience.This whil include detailing the purpose of the study and for whom the results will be of interest. It should also provide enough

\end{document}

\documentclass[10pt,a4paper]{book}

\begin{document}

\small

\begin{flushright}
NET-BASED DISSEMINATION OF E-RESEARCH RESULTS\quad \textbf{189}
\end{flushright}

\begin{flushleft}
\normalsize
\textbf{DISSEMINATION THROUGH PEER-REVIEWED ARTICLES}
\end{flushleft}

The peer-reviewed aeticle has,for over 300 years,been the mark of quality,credence,and acceptability for the academic researcher.This exalted status has arisen not because the process is perfect,but,like democracy,because it may be the best alternative from a set of even less attractive options.But befor we discuss these advantages,we briefly describe the main components of the peer-review process.An understanding of the steps of this rather convoluted process may help e-researchers determine if this means of dissemination is the most appropriate vehicle for thir results.

The editor of the selected peer-review jounal acts as an important first filter for all artical submissions.This person,or a very small committec,generally reviews articles submitted and makes a critically important initial decision.If the article is deemed relevant to the journal's readership and is at least of minimal legibility and comprehensibility and not repetitive of previously published articles,then the editor usually makes a decision to proceed with a full peer review.Often as many as 50 percent of the articles submitted are rejected by the editor at this first point in the review process.The editor then decides on the most qualified persons to review the articles that pass this first hurdle.These persons are usually active researchers who have considerable publishing experience and often serve as members of the journal's editorial review committee.The article is then sent for external review to two to three reviewers,theoretically chosen by the editor for the extent of their expert knowledge that is directly reated to the content of the article.In fact,the pool of prospective reviewers is often limited to those known to the editor and those having a track record of timely return of reviews and whose feedback is not likely to be in radical opposition to the other reviews,which would create more work for the editor.Usually a review assessment form,sent with the article,is used by the reciewers to make their publishing recommendation and provides a space for suggestions and ideas for improvement.The process is double blind in the sense that any references that could identify the author are removed and the reviewers themselves do not identify themselves or the organization with which they are affiliated.Only in rare circumstances are articles accepted without attention to the concerns and suggestions of the reviewers.Normally,based on the reviewer's comments and suggestions,the article is either rejected outright or returned to the author for edits and revisions.The author is expected to address each of these concerns(or defend why they need not be addressed) and return the revised article.In some cases the revisions called for are so extensive as to require a second round of peer review.Otherwise,the revised article is usually reviewed by the editor and then sent for formatting and printing or publication via the networks.

The peer-review process is cumbersome and can be very lengthy.Peer-reviewed journal publication has also been criticized as being narrow (only two to three reviewers see the work),secretive(only the editor knows all actors),arbitrary(the editor may only have a limited number of reviewers to choose from,none of whom may have the necessary knowledge or time to produce a quality review),expensive(the rising cost of academic journals tends to restrict access to only those affiliated with the largest research universities),and slow(even after the lengthy review process,long waits may

\end{document}
